%&pdflatex
\documentclass{article}
\usepackage[final]{nips_2017}
\usepackage[utf8]{inputenc} % allow utf-8 input
\usepackage[T1]{fontenc}    % use 8-bit T1 fonts
\usepackage{hyperref}       % hyperlinks
\usepackage{url}            % simple URL typesetting
\usepackage{booktabs}       % professional-quality tables
\usepackage{amsfonts}       % blackboard math symbols
\usepackage{nicefrac}       % compact symbols for 1/2, etc.
\usepackage{microtype}      % microtypography

\title{Scale Space Edge Detection}

\author{
  Nick Draper, Jonathan Hayase\\
  Seminar in Differential Geometry\\
  Harvey Mudd College
}

\begin{document}
\maketitle

\begin{abstract}
    Scale space representation is the idea that a two dimensional image can be represented by a collection of smoothed images. This paper documents how using such a representation can be useful for detecting edges in an image. The scale space allows for a classification of how strong different edges are in the image from very fine to coarse ones. 
\end{abstract}

\section{Scale Space}
To understand how exactly we detect images in the scale space, we must first define what the scale space is. If we have a continuous function of a variable such as $f(x)$, then we define the scale space representation of such a function as 

\begin{equation}
L(x;t) = g(x;t) * f(x)
\end{equation}

Here $t$ represents the scale parameter, and can be thought of how much smoothing is applied to the function. The function $g$ is the Gaussian kernel given by

\begin{equation}
g(x;t) = \frac{1}{2 \pi t}e^{-(x^2+y^2)/(2t)}
\end{equation}

\section*{References}


\end{document}
